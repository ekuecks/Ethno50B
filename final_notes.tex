\documentclass[]{article}
\usepackage{lmodern}
\usepackage{amssymb,amsmath}
\usepackage{ifxetex,ifluatex}
\usepackage{fixltx2e} % provides \textsubscript
\ifnum 0\ifxetex 1\fi\ifluatex 1\fi=0 % if pdftex
  \usepackage[T1]{fontenc}
  \usepackage[utf8]{inputenc}
\else % if luatex or xelatex
  \ifxetex
    \usepackage{mathspec}
  \else
    \usepackage{fontspec}
  \fi
  \defaultfontfeatures{Ligatures=TeX,Scale=MatchLowercase}
\fi
% use upquote if available, for straight quotes in verbatim environments
\IfFileExists{upquote.sty}{\usepackage{upquote}}{}
% use microtype if available
\IfFileExists{microtype.sty}{%
\usepackage{microtype}
\UseMicrotypeSet[protrusion]{basicmath} % disable protrusion for tt fonts
}{}
\usepackage{hyperref}
\hypersetup{unicode=true,
            pdfborder={0 0 0},
            breaklinks=true}
\urlstyle{same}  % don't use monospace font for urls
\IfFileExists{parskip.sty}{%
\usepackage{parskip}
}{% else
\setlength{\parindent}{0pt}
\setlength{\parskip}{6pt plus 2pt minus 1pt}
}
\setlength{\emergencystretch}{3em}  % prevent overfull lines
\providecommand{\tightlist}{%
  \setlength{\itemsep}{0pt}\setlength{\parskip}{0pt}}
\setcounter{secnumdepth}{0}
% Redefines (sub)paragraphs to behave more like sections
\ifx\paragraph\undefined\else
\let\oldparagraph\paragraph
\renewcommand{\paragraph}[1]{\oldparagraph{#1}\mbox{}}
\fi
\ifx\subparagraph\undefined\else
\let\oldsubparagraph\subparagraph
\renewcommand{\subparagraph}[1]{\oldsubparagraph{#1}\mbox{}}
\fi

\date{}

\begin{document}

\begin{enumerate}
\def\labelenumi{\arabic{enumi}.}
\tightlist
\item
  Extra Credit Paper

  \begin{itemize}
  \tightlist
  \item
    Just focus on one group if you go to the UCLA concert
  \item
    Have fun with it
  \item
    Relate it to class

    \begin{itemize}
    \tightlist
    \item
      Does anything sound familiar
    \end{itemize}
  \item
    Tell about the group (instrumentation, size)

    \begin{itemize}
    \tightlist
    \item
      How long are the songs
    \item
      Does everyone improvise
    \end{itemize}
  \end{itemize}
\item
  Post Bop = After Bop, Cool, and Hard Bop
\end{enumerate}

\section{Avant Garde (Popular
1960s-1970s)}\label{avant-garde-popular-1960s-1970s}

\begin{enumerate}
\def\labelenumi{\arabic{enumi}.}
\tightlist
\item
  More of a change in mindset/approach to music
\item
  Literally ``Out front'', ``Ahead''

  \begin{itemize}
  \tightlist
  \item
    ``New thing'' is less egotistical
  \end{itemize}
\item
  Lasting impact on non avant garde musicians
\item
  Questioning of accepted performance practices -- melody, chords, time,
  etc.
\item
  Developed by Charles Mingus, Max Roach
\item
  Since there is not really a melody or chords, what did composers do

  \begin{itemize}
  \tightlist
  \item
    Not really clear
  \end{itemize}
\item
  Vibrant scene in Chicago
\item
  World Saxophone Quartet (New York)

  \begin{itemize}
  \tightlist
  \item
    Elements of Free
  \item
    2 alto, 1 tenor, 1 bari (compared to traditional 1 soprano instead
    of second alto)
  \end{itemize}
\item
  Make up new changes as the song develops
\item
  During Civil Rights movement

  \begin{itemize}
  \tightlist
  \item
    Musical militance
  \end{itemize}
\end{enumerate}

\section{Free Jazz}\label{free-jazz}

\begin{enumerate}
\def\labelenumi{\arabic{enumi}.}
\tightlist
\item
  Sub genre of Avant Garde
\item
  Music that is not tied to a traditional chord progression
\item
  Spontaneously generated chord progression or none at all
\item
  Tempo could also be free
\item
  Does not have a traditional catchy melody
\item
  Often involves collective improvisation
\item
  Sonic textures are more important than the melody

  \begin{itemize}
  \tightlist
  \item
    Unique blending of instruments
  \item
    Could be playing an instrument in a different way (e.g.~plucking
    piano strings by hand)
  \end{itemize}
\item
  Drummers no longer just kept time, played melody
\item
  Frowned upon by traditional jazz musicians
\item
  Ornette Coleman (Alto/Trumpet/Violin/Composer/Leader, 1930-2015)

  \begin{itemize}
  \tightlist
  \item
    Mainly Alto
  \item
    One of the most influential forces in the Avant Garde (some consider
    him as important as Charlie Parker)
  \item
    Not a virtuoso
  \item
    One of the most prolific composers of the Avant Garde (one of the
    most prolific post Bop composers)
  \item
    The 5 Spot (New York)

    \begin{itemize}
    \tightlist
    \item
      Club popular among artists of the time
    \item
      Ornette played there 4 months, 6 nights per week
    \end{itemize}
  \item
    ``Free Jazz'' Album

    \begin{itemize}
    \tightlist
    \item
      One piece filled both sides of the record
    \end{itemize}
  \end{itemize}
\item
  Don Cherry (Trumpet/Composer/Leader, 1936-1995)

  \begin{itemize}
  \tightlist
  \item
    Important sideman on almost all of Ornette Coleman's earlier
    recordings
  \item
    Influenced by Fats Navarro and Clifford Brown
  \item
    Also influenced by Coleman's musicianship and approach to improv
  \end{itemize}
\item
  Cecil Taylor (Piano/Leader, 1929-Present)

  \begin{itemize}
  \tightlist
  \item
    Created an alternative to modern mainstream piano style
  \item
    ``Wild, turbulent''
  \item
    Does not play with a traditional jazz swing feel
  \item
    Incorporates syncopation
  \end{itemize}
\item
  Albert Ayler (Tenor/Soprano/Vocals/Composer, 1936-1970)

  \begin{itemize}
  \tightlist
  \item
    Original yet unusual improviser
  \item
    considered to have influenced Coltrane's late career
  \item
    Used the entire range of the tenor, utilized altissimo
  \end{itemize}
\item
  Charles Mingus (Bass/Composer/Leader, 1922-1979)

  \begin{itemize}
  \tightlist
  \item
    Combined composition with improvisation while also creating
    accompaniment patterns

    \begin{itemize}
    \tightlist
    \item
      Could alter song during a performance, for instance letting a
      soloist go on longer
    \item
      Multiple background sections to decide in the moment which to play
    \end{itemize}
  \item
    Small big band
  \item
    Second only to Ellington in the complexity of his compositions
  \end{itemize}
\item
  Eric Dolphy (Alto/Flute/Bass Clarinet/Composer, 1928-1964)

  \begin{itemize}
  \tightlist
  \item
    Considered a virtuoso on all 3
  \item
    Also played Tonal jazz

    \begin{itemize}
    \tightlist
    \item
      Featured guest with John Coltrane Quartet
    \end{itemize}
  \item
    Unusual ``mathematical'' element to compositions
  \end{itemize}
\item
  Rahsaan Rowland Kirk (Tenor/Flute/Clarinet, 1936-1977)

  \begin{itemize}
  \tightlist
  \item
    Also played various unusual woodwinds and saxophone derivatives
  \item
    One of the first to explore circle breathing
  \end{itemize}
\item
  Sun Ra (Piano/Composer/Leader, 1915-1993)

  \begin{itemize}
  \tightlist
  \item
    Much older than most avant garde musicians
  \item
    Interest in outer space, thought there was a link between his music
    and outer space
  \item
    Open minded, forward thinking
  \item
    Brought new types of instruments into his ensemble
  \item
    Small big band
  \item
    One of the most innovative orchestrators since Duke Ellington
  \end{itemize}
\item
  AACM - Association for the advancement of creative musicians

  \begin{itemize}
  \tightlist
  \item
    Creative artists incubator
  \item
    Could not financially sustain itself
  \end{itemize}
\item
  Art Ensemble of Chicago

  \begin{itemize}
  \tightlist
  \item
    One of the few Avant Garde ensembles to achieve prominence
  \item
    Performed in African style costumes
  \item
    Played not jazz but great black music (more of an artform)
  \item
    Largest following was white college students in France
  \end{itemize}
\item
  Charlie Haden (Bass, 1937-2014)

  \begin{itemize}
  \tightlist
  \item
    Played with Ornette Coleman's Quartet
  \end{itemize}
\end{enumerate}

\end{document}
