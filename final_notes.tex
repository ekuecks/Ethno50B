\documentclass[]{article}
\usepackage{lmodern}
\usepackage{amssymb,amsmath}
\usepackage{ifxetex,ifluatex}
\usepackage{fixltx2e} % provides \textsubscript
\ifnum 0\ifxetex 1\fi\ifluatex 1\fi=0 % if pdftex
  \usepackage[T1]{fontenc}
  \usepackage[utf8]{inputenc}
\else % if luatex or xelatex
  \ifxetex
    \usepackage{mathspec}
  \else
    \usepackage{fontspec}
  \fi
  \defaultfontfeatures{Ligatures=TeX,Scale=MatchLowercase}
\fi
% use upquote if available, for straight quotes in verbatim environments
\IfFileExists{upquote.sty}{\usepackage{upquote}}{}
% use microtype if available
\IfFileExists{microtype.sty}{%
\usepackage{microtype}
\UseMicrotypeSet[protrusion]{basicmath} % disable protrusion for tt fonts
}{}
\usepackage{hyperref}
\hypersetup{unicode=true,
            pdfborder={0 0 0},
            breaklinks=true}
\urlstyle{same}  % don't use monospace font for urls
\IfFileExists{parskip.sty}{%
\usepackage{parskip}
}{% else
\setlength{\parindent}{0pt}
\setlength{\parskip}{6pt plus 2pt minus 1pt}
}
\setlength{\emergencystretch}{3em}  % prevent overfull lines
\providecommand{\tightlist}{%
  \setlength{\itemsep}{0pt}\setlength{\parskip}{0pt}}
\setcounter{secnumdepth}{0}
% Redefines (sub)paragraphs to behave more like sections
\ifx\paragraph\undefined\else
\let\oldparagraph\paragraph
\renewcommand{\paragraph}[1]{\oldparagraph{#1}\mbox{}}
\fi
\ifx\subparagraph\undefined\else
\let\oldsubparagraph\subparagraph
\renewcommand{\subparagraph}[1]{\oldsubparagraph{#1}\mbox{}}
\fi

\date{}

\begin{document}

\begin{enumerate}
\def\labelenumi{\arabic{enumi}.}
\tightlist
\item
  Extra Credit Paper

  \begin{itemize}
  \tightlist
  \item
    Just focus on one group if you go to the UCLA concert
  \item
    Have fun with it
  \item
    Relate it to class

    \begin{itemize}
    \tightlist
    \item
      Does anything sound familiar
    \end{itemize}
  \item
    Tell about the group (instrumentation, size)

    \begin{itemize}
    \tightlist
    \item
      How long are the songs
    \item
      Does everyone improvise
    \end{itemize}
  \end{itemize}
\item
  Post Bop = After Bop, Cool, and Hard Bop, early 1960s and later
\item
  No period matching section on final (Most don't fit in a single
  period)
\end{enumerate}

\section{Avant Garde (Popular
1960s-1970s)}\label{avant-garde-popular-1960s-1970s}

\begin{enumerate}
\def\labelenumi{\arabic{enumi}.}
\tightlist
\item
  More of a change in mindset/approach to music
\item
  Literally ``Out front'', ``Ahead''

  \begin{itemize}
  \tightlist
  \item
    ``New thing'' is less egotistical
  \end{itemize}
\item
  Lasting impact on non avant garde musicians
\item
  Questioning of accepted performance practices -- melody, chords, time,
  etc.
\item
  Developed by Charles Mingus, Max Roach
\item
  Since there is not really a melody or chords, what did composers do

  \begin{itemize}
  \tightlist
  \item
    Not really clear
  \end{itemize}
\item
  Vibrant scene in Chicago
\item
  World Saxophone Quartet (New York)

  \begin{itemize}
  \tightlist
  \item
    Elements of Free
  \item
    2 alto, 1 tenor, 1 bari (compared to traditional 1 soprano instead
    of second alto)
  \end{itemize}
\item
  Make up new changes as the song develops
\item
  During Civil Rights movement

  \begin{itemize}
  \tightlist
  \item
    Musical militancy
  \end{itemize}
\item
  Piano Trio

  \begin{itemize}
  \tightlist
  \item
    Piano, Drums, Bass
  \end{itemize}
\item
  Bill Evans (Piano/Composer/Leader, 1929-1980)

  \begin{itemize}
  \tightlist
  \item
    Rose to prominence in late 1950s as side man with Miles Davis on
    ``Kind of Blue''
  \item
    Viewed as most influential pianist since Bud Powell
  \item
    Hard to categorize
  \item
    One of the most influential Post-Bop pianist
  \item
    Style based in Cool, developed his own original approach
  \item
    Made use of new chord voices and modal harmony

    \begin{itemize}
    \tightlist
    \item
      Chord voice - How the chord is constructed, chords usually based
      in thirds
    \item
      Bill Evans built chords in fourths, less determined major
      vs.~minor
    \end{itemize}
  \item
    Redefined the piano trio

    \begin{itemize}
    \tightlist
    \item
      Traditional roles of instruments were changed
    \item
      Sometimes Bass would keep time/play melody
    \item
      Emphasis of the beat was not present though time was still steady
    \item
      ``Floating'' style
    \item
      Avoided ``obvious'' accent points like first beat of a measure
    \item
      Used 3/4 time much more frequently than most
    \end{itemize}
  \item
    Bill Evans trio

    \begin{itemize}
    \tightlist
    \item
      Scott LaFaro (Bass)
    \item
      Paul Motian (Drums)
    \end{itemize}
  \item
    Adapted harmonies from Impressionist Classical composers

    \begin{itemize}
    \tightlist
    \item
      larger chords (4,5+ notes)
    \end{itemize}
  \item
    ``Nardis''
  \end{itemize}
\item
  McCoy Tyner (Piano/Composer/Leader, 1938-Present)

  \begin{itemize}
  \tightlist
  \item
    Still playing
  \item
    One of the most influential piano players of the 21st century
  \item
    Got his start with Benny Golson and Art Farmer's jazztet in 1960
  \item
    Part of Coltrane's Quartet later in 1960

    \begin{itemize}
    \tightlist
    \item
      Left in 1965 when Coltrane's music was becoming atonal and free

      \begin{enumerate}
      \def\labelenumii{\roman{enumii}.}
      \tightlist
      \item
        More percussion musicians were added
      \item
        Tyner didn't have any feelings toward the music, just ``noise''
        so left
      \end{enumerate}
    \end{itemize}
  \item
    Strong left hand
  \end{itemize}
\item
  Ahmad Jamal (Piano/Composer/Leader, 1930-Present)

  \begin{itemize}
  \tightlist
  \item
    Influence on Miles Davis, particularly his use of space
  \item
    Known for incorporating vamps - repeating rhythmic figure, usually
    with a melodic component
  \item
    ``Ponciana''
  \end{itemize}
\item
  Cedar Walton (Piano/Composer/Leader, 1934-2013)

  \begin{itemize}
  \tightlist
  \item
    Member of Jazztet, Jazz Messengers, briefly with John Coltrane
    (Giant Steps album), played with Charley Harrison
  \item
    Started as a classical pianist
  \end{itemize}
\item
  Herbie Hancock (Piano/Composer/Leader/Keyboardist, 1940-Present)

  \begin{itemize}
  \tightlist
  \item
    Important to Jazz Fusion
  \item
    Sideman of Miles Davis Second Great Quintet
  \item
    Started career as child prodigy
  \item
    Performed with Chicago Symphony as a featured soloist at age 11

    \begin{itemize}
    \tightlist
    \item
      Even more notable for an African American to do so in 1951
    \end{itemize}
  \item
    Refined the Bill Evans style
  \item
    Influenced by Duke Ellington, Oscar Peterson, George Shearing
  \item
    ``Maiden Voyage'' - first album
  \item
    Pioneer in use of electronics
  \item
    Won Album of the Year Grammy in 2007
  \item
    Fusion band - The Headhunters
  \end{itemize}
\item
  Chick Corea (Piano/Composer/Leader/Keyboard, 1941-Present)

  \begin{itemize}
  \tightlist
  \item
    Influenced by Horace Silver, Bill Evans, Bud Powell, Thelonius Monk,
    McCoy Tyner
  \item
    Classical influence as well
  \item
    Crisp, staccato style
  \item
    Spanish, Latin American themes
  \item
    Modern chord voicings
  \item
    Fusion band - Return to Forever

    \begin{itemize}
    \tightlist
    \item
      Akoustic and Electrik bands
    \end{itemize}
  \item
    Sideman for Miles Davis in late 1960's (start of Fusion)
  \end{itemize}
\item
  Keith Jarrett (Piano/Composer/Leader, 1945-Present)

  \begin{itemize}
  \tightlist
  \item
    Child prodigy
  \item
    Played briefly with the Jazz Messengers
  \item
    Briefly sideman with Miles Davis
  \item
    Influenced by Bill Davis, Ornette Coleman
  \item
    Avoided staccato type passages
  \item
    Blend of funk, country, world music styles
  \item
    Most successful solo concert pianist in Jazz history

    \begin{itemize}
    \tightlist
    \item
      First one to play in concert halls
    \end{itemize}
  \end{itemize}
\end{enumerate}

\section{Free Jazz}\label{free-jazz}

\begin{enumerate}
\def\labelenumi{\arabic{enumi}.}
\tightlist
\item
  Sub genre of Avant Garde
\item
  Music that is not tied to a traditional chord progression
\item
  Spontaneously generated chord progression or none at all
\item
  Tempo could also be free
\item
  Does not have a traditional catchy melody
\item
  Often involves collective improvisation
\item
  Sonic textures are more important than the melody

  \begin{itemize}
  \tightlist
  \item
    Unique blending of instruments
  \item
    Could be playing an instrument in a different way (e.g.~plucking
    piano strings by hand)
  \end{itemize}
\item
  Drummers no longer just kept time, played melody
\item
  Frowned upon by traditional jazz musicians
\item
  Ornette Coleman (Alto/Trumpet/Violin/Composer/Leader, 1930-2015)

  \begin{itemize}
  \tightlist
  \item
    Mainly Alto
  \item
    One of the most influential forces in the Avant Garde (some consider
    him as important as Charlie Parker)
  \item
    Not a virtuoso
  \item
    One of the most prolific composers of the Avant Garde (one of the
    most prolific post Bop composers)
  \item
    The 5 Spot (New York)

    \begin{itemize}
    \tightlist
    \item
      Club popular among artists of the time
    \item
      Ornette played there 4 months, 6 nights per week
    \end{itemize}
  \item
    ``Free Jazz'' Album

    \begin{itemize}
    \tightlist
    \item
      One piece filled both sides of the record
    \end{itemize}
  \end{itemize}
\item
  Don Cherry (Trumpet/Composer/Leader, 1936-1995)

  \begin{itemize}
  \tightlist
  \item
    Important sideman on almost all of Ornette Coleman's earlier
    recordings
  \item
    Influenced by Fats Navarro and Clifford Brown
  \item
    Also influenced by Coleman's musicianship and approach to improv
  \end{itemize}
\item
  Cecil Taylor (Piano/Leader, 1929-Present)

  \begin{itemize}
  \tightlist
  \item
    Created an alternative to modern mainstream piano style
  \item
    ``Wild, turbulent''
  \item
    Does not play with a traditional jazz swing feel
  \item
    Incorporates syncopation
  \item
    ``Pure energy'' fueled his music
  \item
    Would have ``imaginary concerts'' at night to a pretend audience in
    his apartment
  \item
    Since he prepared for his concerts, the audience should too
  \end{itemize}
\item
  Albert Ayler (Tenor/Soprano/Vocals/Composer, 1936-1970)

  \begin{itemize}
  \tightlist
  \item
    Original yet unusual improviser
  \item
    considered to have influenced Coltrane's late career
  \item
    Used the entire range of the tenor, utilized altissimo
  \end{itemize}
\item
  Charles Mingus (Bass/Composer/Leader, 1922-1979)

  \begin{itemize}
  \tightlist
  \item
    Combined composition with improvisation while also creating
    accompaniment patterns

    \begin{itemize}
    \tightlist
    \item
      Could alter song during a performance, for instance letting a
      soloist go on longer
    \item
      Multiple background sections to decide in the moment which to play
    \end{itemize}
  \item
    Small big band
  \item
    Second only to Ellington in the complexity of his compositions
  \end{itemize}
\item
  Eric Dolphy (Alto/Flute/Bass Clarinet/Composer, 1928-1964)

  \begin{itemize}
  \tightlist
  \item
    Considered a virtuoso on all 3
  \item
    Also played Tonal jazz

    \begin{itemize}
    \tightlist
    \item
      Featured guest with John Coltrane Quartet
    \end{itemize}
  \item
    Unusual ``mathematical'' element to compositions
  \end{itemize}
\item
  Rahsaan Rowland Kirk (Tenor/Flute/Clarinet, 1936-1977)

  \begin{itemize}
  \tightlist
  \item
    Also played various unusual woodwinds and saxophone derivatives
  \item
    One of the first to explore circle breathing
  \end{itemize}
\item
  Sun Ra (Piano/Composer/Leader, 1915-1993)

  \begin{itemize}
  \tightlist
  \item
    Much older than most avant garde musicians
  \item
    Interest in outer space, thought there was a link between his music
    and outer space
  \item
    Open minded, forward thinking
  \item
    Brought new types of instruments into his ensemble
  \item
    Small big band
  \item
    One of the most innovative orchestrators since Duke Ellington
  \end{itemize}
\item
  AACM - Association for the Advancement of Creative Musicians

  \begin{itemize}
  \tightlist
  \item
    Creative artists incubator
  \item
    Could not financially sustain itself
  \end{itemize}
\item
  Art Ensemble of Chicago

  \begin{itemize}
  \tightlist
  \item
    One of the few Avant Garde ensembles to achieve prominence
  \item
    Performed in African style costumes
  \item
    Played not jazz but great black music (more of an art form)
  \item
    Largest following was white college students in France
  \end{itemize}
\item
  Charlie Haden (Bass, 1937-2014)

  \begin{itemize}
  \tightlist
  \item
    Played with Ornette Coleman's Quartet
  \end{itemize}
\end{enumerate}

\section{Jazz Fusion (Popular 1970s)}\label{jazz-fusion-popular-1970s}

\begin{enumerate}
\def\labelenumi{\arabic{enumi}.}
\tightlist
\item
  Mixture of jazz, rock, and funk

  \begin{itemize}
  \tightlist
  \item
    Similar to Hard Bop in that respect

    \begin{itemize}
    \tightlist
    \item
      Not trying to force the other styles into jazz though, rather
      changing jazz to fit in with the others
    \end{itemize}
  \end{itemize}
\item
  Differences in jazz, rock, and funk

  \begin{itemize}
  \tightlist
  \item
    Jazz tends to have longer phrase lengths than rock or funk
  \item
    Frequency of chord changes (Jazz tends to be faster)
  \item
    Jazz tends to be more harmonically complex
  \item
    Jazz tends to have more improvisation

    \begin{itemize}
    \tightlist
    \item
      Biggest contribution from jazz to fusion is improv
    \end{itemize}
  \item
    Jazz tends to have less repetition of drum beats
  \item
    Jazz tends to have less repetition of bass lines
  \end{itemize}
\item
  Fusion musicians grew up in the 1950s/60s when rock was popular
\item
  Origin

  \begin{itemize}
  \tightlist
  \item
    Improvisational style of Coltrane and Tyner
  \item
    Accompanying styles of Pop and R\&B bands
  \item
    Songs based on vamps
  \item
    Popularity of the electric guitar, bass (Fender bass) and keyboard
  \item
    Popularity/desire to bring in auxiliary percussion instruments
  \end{itemize}
\item
  Guitar had a leading role in most bands
\item
  ``Older'' generation of guitarists

  \begin{itemize}
  \tightlist
  \item
    John McLaughlin (Guitar/Leader, 1942-Present)

    \begin{itemize}
    \tightlist
    \item
      solid body guitar
    \item
      Virtuoso
    \item
      First to take few breaks in his solo\\
    \item
      One of the most prominent guitarists since Wes Montgomery
    \item
      Leader of Mahavishnu orchestra (5 members)

      \begin{enumerate}
      \def\labelenumii{\roman{enumii}.}
      \tightlist
      \item
        Double Necked Guitar, Keyboard, Bass, Drums, Violin
      \end{enumerate}
    \end{itemize}
  \item
    Larry Coryell (Guitar, 1943-Present)

    \begin{itemize}
    \tightlist
    \item
      From Texas
    \item
      Fused Jazz with Country
    \item
      Equally rooted in Rock and Hard Bop
    \item
      Classically trained as well
    \item
      Hollow body guitar
    \end{itemize}
  \item
    Pat Metheny (Guitar/Leader, 1954-Present)

    \begin{itemize}
    \tightlist
    \item
      Most influential Fusion guitarist
    \item
      Notably used electronic effects (chorus effect)

      \begin{enumerate}
      \def\labelenumii{\roman{enumii}.}
      \tightlist
      \item
        Slightly delayed, slightly out of tune version of each pitch
        superimposed on the original
      \end{enumerate}
    \item
      First album ``Bright Size Life''
    \item
      Pat Metheny Trio - Guitar, Bass, Drums
    \item
      Hollow body guitar
    \end{itemize}
  \end{itemize}
\item
  ``Newer'' generation of guitarists

  \begin{itemize}
  \tightlist
  \item
    Hiram Bullock (Guitar, 1955-2008)

    \begin{itemize}
    \tightlist
    \item
      Originally a saxophonist
    \item
      Member of the Late Show David Letterman band in the 1980s
    \item
      Sideman for Miles Davis
    \item
      Solid body Fender Stratocaster (same guitar as Jimi Hendrix)
    \end{itemize}
  \item
    John Scofield (Guitar, 1951-Present)

    \begin{itemize}
    \tightlist
    \item
      Played with Miles Davis, Joe Henderson, Herbie Hancock, Pat
      Metheny among others
    \item
      Bass Desires
    \item
      Arguably equally skilled in Bop, Fusion, Funk, Soul
    \item
      Semi-hollow body guitar
    \end{itemize}
  \item
    Mike Stern (Guitar/Leader, 1953-Present)

    \begin{itemize}
    \tightlist
    \item
      Played with Miles Davis
    \item
      5 time Grammy nominee
    \item
      Solid body guitar
    \item
      Chorus effect
    \end{itemize}
  \end{itemize}
\item
  Joe Zawinul (Piano, 1932-2007)

  \begin{itemize}
  \tightlist
  \item
    One of the first to utilize keyboard
  \item
    Came to prominence playing with Cannonball Adderly
  \item
    Co-founded the groups Weather Report

    \begin{itemize}
    \tightlist
    \item
      One of the leading fusion bands
    \item
      Co-founded by Wayne Shorter (Tenor Sax), Miroslav Vitous and
      Zawinul
    \item
      Shorter and Zawinul met in Maynard Ferguson's band
    \item
      ``Birdland'' on our listening list
    \end{itemize}
  \item
    Co-founded the Zawinul Syndicate
  \end{itemize}
\item
  Jaco Pastorius (Bass, 1951-1987)

  \begin{itemize}
  \tightlist
  \item
    Joined Weather Report in 1976
  \item
    First virtuoso electric bassist
  \item
    Played fret-less bass

    \begin{itemize}
    \tightlist
    \item
      Changes the sound of the instrument
    \item
      Makes the sound richer and warmer, more like an upright bass
    \end{itemize}
  \end{itemize}
\end{enumerate}

\end{document}
