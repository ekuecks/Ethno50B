\documentclass[]{article}
\usepackage{lmodern}
\usepackage{amssymb,amsmath}
\usepackage{ifxetex,ifluatex}
\usepackage{fixltx2e} % provides \textsubscript
\ifnum 0\ifxetex 1\fi\ifluatex 1\fi=0 % if pdftex
  \usepackage[T1]{fontenc}
  \usepackage[utf8]{inputenc}
\else % if luatex or xelatex
  \ifxetex
    \usepackage{mathspec}
  \else
    \usepackage{fontspec}
  \fi
  \defaultfontfeatures{Ligatures=TeX,Scale=MatchLowercase}
\fi
% use upquote if available, for straight quotes in verbatim environments
\IfFileExists{upquote.sty}{\usepackage{upquote}}{}
% use microtype if available
\IfFileExists{microtype.sty}{%
\usepackage{microtype}
\UseMicrotypeSet[protrusion]{basicmath} % disable protrusion for tt fonts
}{}
\usepackage{hyperref}
\hypersetup{unicode=true,
            pdfborder={0 0 0},
            breaklinks=true}
\urlstyle{same}  % don't use monospace font for urls
\IfFileExists{parskip.sty}{%
\usepackage{parskip}
}{% else
\setlength{\parindent}{0pt}
\setlength{\parskip}{6pt plus 2pt minus 1pt}
}
\setlength{\emergencystretch}{3em}  % prevent overfull lines
\providecommand{\tightlist}{%
  \setlength{\itemsep}{0pt}\setlength{\parskip}{0pt}}
\setcounter{secnumdepth}{0}
% Redefines (sub)paragraphs to behave more like sections
\ifx\paragraph\undefined\else
\let\oldparagraph\paragraph
\renewcommand{\paragraph}[1]{\oldparagraph{#1}\mbox{}}
\fi
\ifx\subparagraph\undefined\else
\let\oldsubparagraph\subparagraph
\renewcommand{\subparagraph}[1]{\oldsubparagraph{#1}\mbox{}}
\fi

\date{}

\begin{document}

\section{General Knowledge}\label{general-knowledge}

\begin{enumerate}
\def\labelenumi{\arabic{enumi}.}
\tightlist
\item
  harmony = chord progression = chord changes = chords = changes
\item
  Jazz musicians tend to not follow a particular movement, but rather
  their own interests

  \begin{itemize}
  \tightlist
  \item
    They tend to dislike labels assigned to them
  \end{itemize}
\item
  Side = Individual song on a record
\end{enumerate}

\section{Bop (1940s - Present)}\label{bop-1940s---present}

\begin{enumerate}
\def\labelenumi{\arabic{enumi}.}
\tightlist
\item
  Followed the Swing Era
\item
  Was a rebellion against the commercialization of jazz that had
  occurred
\item
  Started with jam sessions where musicians could branch out and be more
  creative
\item
  Combos more popular than big bands
\item
  Charlie Parker (Alto, 1920-1955)

  \begin{itemize}
  \tightlist
  \item
    Most influential Bop musician
  \item
    Unique approach to improvisation

    \begin{itemize}
    \tightlist
    \item
      Used dissonance well
    \end{itemize}
  \item
    From Kansas City, Missouri (town known for blues)
  \item
    Married at 16 (soon divorced to move to NY)
  \item
    Became addicted to heroin at 17
  \item
    Known as ``Bird''
  \item
    Later in life was committed to a mental hospital after passing out
    and setting his bed on fire
  \item
    After getting out of the hospital, he formed a quartet containing
    Miles Davis
  \item
    Hated the term ``Bebop'' -- ``It's just music''
  \item
    Very critical of his own work and shy about the praise he received
  \item
    Loved all kinds of music

    \begin{itemize}
    \tightlist
    \item
      Known to listen to Country music
    \end{itemize}
  \end{itemize}
\item
  John Birks (Dizzy) Gillespie (Trumpet, 1917-1993)

  \begin{itemize}
  \tightlist
  \item
    Most influential Bop trumpet player
  \item
    Known for great range
  \item
    Name comes from his unpredictability

    \begin{itemize}
    \tightlist
    \item
      Would often dance during others' solos
    \end{itemize}
  \item
    Public face of Bop
  \item
    Broke many barriers

    \begin{itemize}
    \tightlist
    \item
      Hired a female trombonist and a Cuban bongo player for his band
    \end{itemize}
  \item
    Tied in jazz and Caribbean music
  \item
    Tried to make Bop accessible to everyone

    \begin{itemize}
    \tightlist
    \item
      Failed to attract dancers
    \end{itemize}
  \end{itemize}
\item
  Thelonius Monk (Piano, 1917-1982)

  \begin{itemize}
  \tightlist
  \item
    Not very technically proficient
  \item
    Unique harmonic approach
  \item
    Embraced dissonance
  \item
    ``Logical''
  \item
    Resident pianist at Minton's
  \item
    Eccentric: Would dress oddly, speak very little, get up during a
    performance to dance

    \begin{itemize}
    \tightlist
    \item
      caused critics to dislike/dismiss him
    \end{itemize}
  \item
    Denied a cabaret card after refusing to testify against Bud Powell
    for possession of narcotics

    \begin{itemize}
    \tightlist
    \item
      Means he could not play in venues that sold alcohol
    \item
      Spent 6 years writing music
    \item
      Released an album that received great reviews which sparked new
      popularity for him
    \end{itemize}
  \end{itemize}
\item
  Tadd Dameron (Piano/Composer, 1917-1965)

  \begin{itemize}
  \tightlist
  \item
    Wrote many modern day jazz standards
  \item
    Wrote Hot House (Our recording is by Dizzy Gillespie)
  \item
    Led larger groups (7-10 musicians)
  \end{itemize}
\item
  John Lewis (Piano/Composer, 1920-2001)

  \begin{itemize}
  \tightlist
  \item
    Spanned many periods
  \item
    Modern Jazz Quartet founder (Late in Bop, persisted through Cool)

    \begin{itemize}
    \tightlist
    \item
      Former members of Dizzy Gillespie's band
    \item
      Played in concert halls and wore tuxedos
    \item
      Piano, Drums, Bass, and Vibes
    \end{itemize}
  \item
    Loathed the dissipation and drug use of Charlie Parker
  \item
    Insisted that his music be played with dignity
  \end{itemize}
\item
  Kenny Clarke (Drums, 1914-1985)

  \begin{itemize}
  \tightlist
  \item
    Changed the way that drummers played, invented Bop style
  \item
    Drummer in house band at Minton's Playhouse
  \item
    Kept time on the cymbal
  \end{itemize}
\item
  Minton's Playhouse

  \begin{itemize}
  \tightlist
  \item
    Location for after hours jam sessions that gave rise to the Bop
    period
  \end{itemize}
\item
  Max Roach (Drums, 1924-2007)

  \begin{itemize}
  \tightlist
  \item
    Also known for hard Bop period
  \end{itemize}
\item
  Charles Mingus (Bass/Composer, 1922-1979)

  \begin{itemize}
  \tightlist
  \item
    Spanned multiple periods
  \end{itemize}
\item
  Ray Brown (Bass, 1926-2002)
\item
  Oscar Pettiford (Bass/Cello, 1922-1960)
\item
  Bud Powell (Piano, 1924-1966)

  \begin{itemize}
  \tightlist
  \item
    Brought Bop to the Keyboard
  \item
    ``Could outbird Bird and outdizzy Dizzy''
  \item
    Debatably a better improviser than Charlie Parker
  \end{itemize}
\item
  George Shearing (Piano, 1919-2011)

  \begin{itemize}
  \tightlist
  \item
    Commercially popular
  \item
    Worked with Mel Torme
  \end{itemize}
\item
  Oscar Peterson (Piano, 1925-2007)

  \begin{itemize}
  \tightlist
  \item
    One of the most admired pianists ever
  \item
    Incredibly technical
  \end{itemize}
\item
  Dexter Gordon (Tenor, 1923-1990)

  \begin{itemize}
  \tightlist
  \item
    Bop Pioneer for tenor sax
  \end{itemize}
\item
  Stan Getz (Tenor, 1927-1991)

  \begin{itemize}
  \tightlist
  \item
    Spanned multiple periods

    \begin{itemize}
    \tightlist
    \item
      Cool, Bossa Nova
    \end{itemize}
  \item
    Member of Woody Herman's Thundering Herd
  \end{itemize}
\item
  Sonny Stitt (Alto/Tenor, 1924-1982)

  \begin{itemize}
  \tightlist
  \item
    Criticized for copying Charlie Parker
  \end{itemize}
\item
  Fats Navarro (Trumpet, 1923-1950)
\item
  J.J. Johnson (Trombone, 1924-2001)
\item
  Woody Herman's Thundering Herd

  \begin{itemize}
  \tightlist
  \item
    ``Four Brothers''

    \begin{itemize}
    \tightlist
    \item
      Named for the 4 saxophone players in the band
    \end{itemize}
  \item
    Big Band with some Bop music
  \end{itemize}
\item
  Primarily developed in New York
\item
  Primarily African-American
\end{enumerate}

\section{Cool (Late 1940s - Present)}\label{cool-late-1940s---present}

\begin{enumerate}
\def\labelenumi{\arabic{enumi}.}
\tightlist
\item
  Easier to follow and softer than Bop
\item
  May be viewed as a reaction to Bop (Harrison doesn't think so though)
\item
  Primarily developed in L.A./west coast (vs.~NY for Bop)
\item
  Primarily Caucasian (vs.~African-American for Bop)
\item
  Synonymous with ``West Coast Sound''

  \begin{itemize}
  \tightlist
  \item
    Really West Coast Sound contained in Cool
  \end{itemize}
\item
  Dry, light, airy sound compared to harsher, stronger Bop sound
\item
  Lester Young (Tenor, 1909-1959)

  \begin{itemize}
  \tightlist
  \item
    Influenced both Bop and Cool
  \item
    Known as a great swing era improviser
  \item
    Played with Basie
  \item
    Played at Minton's
  \item
    Used dissonance to create new harmonies in solos (influenced Bop)
  \item
    Legato phrasing (influenced Cool)
  \item
    Uncharacteristic sound

    \begin{itemize}
    \tightlist
    \item
      Most other tenors at the time gravitated to the lower register
    \item
      Lester approached it more like an Alto player would
    \item
      Delicate, light
    \end{itemize}
  \end{itemize}
\item
  Miles Davis (Trumpet, 1926-1991)

  \begin{itemize}
  \tightlist
  \item
    Disliked the Cool label
  \item
    New York musician
  \item
    Album: ``Birth of the Cool'' (1951)

    \begin{itemize}
    \tightlist
    \item
      Often mistakenly assumed to be the start of Cool
    \item
      First of many collaborations with composer/pianist Ernest Gilmore
      ``Gil'' Evans
    \item
      10 piece group
    \item
      Notably contained tuba, bass clarinet, french horn, flute
    \item
      Lee Konitz as a side man
    \end{itemize}
  \end{itemize}
\item
  Stan Getz (Tenor, 1927-1991)

  \begin{itemize}
  \tightlist
  \item
    Disliked the Cool label
  \end{itemize}
\item
  Modern Jazz Quartet (MJQ)

  \begin{itemize}
  \tightlist
  \item
    Disliked the Cool label
  \end{itemize}
\item
  Lennie Tristano (Composer/Band Leader/Piano, 1919-1978)

  \begin{itemize}
  \tightlist
  \item
    Largely overlooked
  \item
    One of the main creators of Cool as a modern jazz alternative to Bop
  \item
    Influenced by pianist Art Tatum and Lester Young
  \item
    Also influence by Bach
  \item
    Less ``jumpy'' than Bop, smoother

    \begin{itemize}
    \tightlist
    \item
      Not necessarily melodic though
    \end{itemize}
  \item
    Often overlooked because he did not release many records
  \item
    Employed collective improvisation

    \begin{itemize}
    \tightlist
    \item
      Multiple soloists at once
    \end{itemize}
  \item
    Influence on Bill Evans (piano)
  \item
    Nearly blind
  \item
    Played with splayed fingers on right hand, but curved left
  \end{itemize}
\item
  Lee Konitz (Alto, 1927-Present)

  \begin{itemize}
  \tightlist
  \item
    Was a student of Lennie Tristano

    \begin{itemize}
    \tightlist
    \item
      Later colleague and band mate
    \end{itemize}
  \item
    Some claim he could rival Charlie Parker

    \begin{itemize}
    \tightlist
    \item
      Very different sound
    \item
      Konitz was dry, airy, light vs.~Parker's brittle sound
    \end{itemize}
  \item
    Preferred upper register
  \end{itemize}
\item
  Big bands had mostly disappeared by this time

  \begin{itemize}
  \tightlist
  \item
    Some survivor's were Woody Herman's and Stan Kenton's bands

    \begin{itemize}
    \tightlist
    \item
      Both moved to the West Coast
    \item
      Many noteworthy musicians started as members of these bands before
      contonuing on to solo careers
    \end{itemize}
  \item
    Transition from dance bands to performance orchestra for a seated
    audience

    \begin{itemize}
    \tightlist
    \item
      More features and solos
    \item
      More tempo changes (taboo for dance bands)
    \end{itemize}
  \end{itemize}
\item
  Warne Marsh (Tenor, 1927-1987)

  \begin{itemize}
  \tightlist
  \item
    Protégé of Lee Konitz
  \end{itemize}
\item
  Chet Baker (Trumpet/Vocalist, 1929-1988)

  \begin{itemize}
  \tightlist
  \item
    Notably did not start in Woody Herman or Stan Kenton's band
  \item
    Started with Jerry Mulligan's (Bari sax player) quartet
  \end{itemize}
\item
  Shorty Rogers (Trumpet/Composer, 1924-1994)

  \begin{itemize}
  \tightlist
  \item
    Played with and wrote for both Woody Herman and Stan Kenton
  \end{itemize}
\item
  Jack Sheldon (Trumpet/Vocalist, 1931-Present)

  \begin{itemize}
  \tightlist
  \item
    Also an actor
  \item
    Sang for School House Rock (``I'm Just a Bill'' among others)
  \end{itemize}
\item
  Art Farmer (Trumpet/Flugelhorn, 1928-1999)

  \begin{itemize}
  \tightlist
  \item
    Originally from Iowa
  \item
    Early teens in Phoenix
  \item
    Started with Lionel Hampton
  \end{itemize}
\item
  Art Pepper (Alto, 1925-1982)

  \begin{itemize}
  \tightlist
  \item
    Started with Stan Kenton
  \end{itemize}
\item
  Jimmy Giuffre (Tenor/Clarinet/Composer, 1921-2008)

  \begin{itemize}
  \tightlist
  \item
    Wrote for Woody Herman

    \begin{itemize}
    \tightlist
    \item
      Wrote ``Four Brothers''
    \end{itemize}
  \end{itemize}
\item
  Paul ``Dry Martini'' Desmond (Alto, 1924-1977)

  \begin{itemize}
  \tightlist
  \item
    Notably did not play with Woody Herman or Stan Kenton
  \item
    Played in Dave Brubeck quartet (only horn player)
  \item
    Similar sound to Lee Konitz
  \end{itemize}
\item
  Gerry Mulligan (Bari/Composer, 1927-1996)

  \begin{itemize}
  \tightlist
  \item
    Sideman on Miles Davis' ``Birth of the Cool''

    \begin{itemize}
    \tightlist
    \item
      As a result, offered a residency at an L.A. club causing him to
      move from New York
    \end{itemize}
  \item
    Uncommon combo instrument
  \item
    Brought the lighter sound of Cool to the Bari
  \item
    Jerry Mulligan's Quartet

    \begin{itemize}
    \tightlist
    \item
      Notably did not have a piano (club probably didn't have one)
    \item
      Trumpet, Bari, Bass, Drums
    \end{itemize}
  \end{itemize}
\item
  Chico Hamilton (Drums, 1921-2013)

  \begin{itemize}
  \tightlist
  \item
    One of the relatively few African-American Cool musicians
  \item
    Played for Stan Kenton's Big Band and Gerry Mulligan's Quartet
  \item
    Ventured away from the ride rhythm
  \end{itemize}
\item
  Shelly Manne (Drums, 1921-2008)

  \begin{itemize}
  \tightlist
  \item
    Played with both Stan Kenton and Woody Herman
  \item
    Drummer for Henry Mancini (TV and Film Composer) (Pink Panther and
    more)
  \end{itemize}
\item
  Jim Hall (Guitar, 1930-2013)

  \begin{itemize}
  \tightlist
  \item
    Mellow sound
  \item
    Did not play very technically
  \end{itemize}
\item
  Carl Fontana (Trombone, 1928-2003)

  \begin{itemize}
  \tightlist
  \item
    Played with Woody Herman
  \item
    Considered second only to J.J. Johnson
  \item
    Could be considered a Bop musician
  \end{itemize}
\item
  Frank Rosolino (Trombone, 1926-1978)

  \begin{itemize}
  \tightlist
  \item
    Could be considered a Bop musician
  \item
    The top west coast trombone player
  \item
    Played with Stan Kenton

    \begin{itemize}
    \tightlist
    \item
      Some compositions were named after him (``Frank Speaking'')
    \end{itemize}
  \end{itemize}
\item
  Bob Brookmeyer (Trombone/Composer, 1929-2011)

  \begin{itemize}
  \tightlist
  \item
    Valve trombone

    \begin{itemize}
    \tightlist
    \item
      Easy way out to avoid the difficulties of a slide
    \end{itemize}
  \end{itemize}
\item
  Dave Brubeck (Piano/Composer, 1920-2012)

  \begin{itemize}
  \tightlist
  \item
    Leader of Dave Brubeck quartet
  \item
    Most commercially successful Cool pianist
  \item
    Popular among college students

    \begin{itemize}
    \tightlist
    \item
      Toured college campuses
    \item
      ``Jazz Goes to College'' album
    \end{itemize}
  \item
    ``Time Out'' (our listening list)

    \begin{itemize}
    \tightlist
    \item
      Unusual time signatures
    \item
      Take 5 (5/4)
    \item
      Blue Rondo a la Turk (9/8)

      \begin{enumerate}
      \def\labelenumii{\roman{enumii}.}
      \tightlist
      \item
        Blue -\textgreater{} Early days of jazz
      \item
        Rondo -\textgreater{} Form of the song, alternating sessions but
        always returning to the original. ABACAD\ldots{}
      \item
        a la Turk -\textgreater{} Tribute to Turkish musicians (Brubeck
        visited Turkey)
      \end{enumerate}
    \end{itemize}
  \end{itemize}
\end{enumerate}

\section{Bossa Nova (Popular
1958-1963)}\label{bossa-nova-popular-1958-1963}

\begin{enumerate}
\def\labelenumi{\arabic{enumi}.}
\tightlist
\item
  Incorporation of Brazilian style
\item
  ``Bossa Nova'' = ``New Trend''/``New Beat''
\item
  Antonio Carlos Jobim (Composer, 1927-1994)

  \begin{itemize}
  \tightlist
  \item
    Most popular Bossa Nova composer
  \item
    Worked with Stan Getz
  \item
    Very important in bringing Bossa Nova to the U.S.
  \item
    ``Desafinado''

    \begin{itemize}
    \tightlist
    \item
      From album ``Getz/Gilberto''
    \end{itemize}
  \end{itemize}
\end{enumerate}

\section{Hard Bop (Mid 1950s-Present)}\label{hard-bop-mid-1950s-present}

\begin{enumerate}
\def\labelenumi{\arabic{enumi}.}
\tightlist
\item
  Evolution from Bop, emerged on East coast
\item
  Some say reaction to Cool
\item
  The Jazz Messengers

  \begin{itemize}
  \tightlist
  \item
    Created by Art Blakey and Horace Silver
  \item
    Many top musicians got their start in this band
  \end{itemize}
\item
  Style invented by African-Americans that white people can't copy

  \begin{itemize}
  \tightlist
  \item
    Rooted in black culture and lifestyle
  \end{itemize}
\item
  Julian ``Cannonball'' Adderly (Alto, 1928-1975)

  \begin{itemize}
  \tightlist
  \item
    Considered most influential saxophonist since Charlie Parker
  \item
    Powerful but fluid sound
  \item
    Sideman with Miles Davis
  \item
    Later co-ran Cannonball Adderly Quartet with his brother Nat Adderly
  \end{itemize}
\item
  Sonny Rollins (Tenor/Composer, 1930-Present)

  \begin{itemize}
  \tightlist
  \item
    The titan of jazz
  \item
    Initially influenced by Charlie Parker
  \item
    Stopped performing in the late 62s because he felt that he was not
    growing as a musician
  \item
    Isolated himself
  \item
    Would go to the Williamsburg bridge in New York to practice

    \begin{itemize}
    \tightlist
    \item
      Artistic rediscovery
    \item
      Released ``The Bridge''

      \begin{enumerate}
      \def\labelenumii{\roman{enumii}.}
      \tightlist
      \item
        ``Without a Song'' on our listening list
      \end{enumerate}
    \end{itemize}
  \item
    Became addicted to heroin

    \begin{itemize}
    \tightlist
    \item
      Worked as a day laborer for a year to get himself clean
    \end{itemize}
  \item
    ``Saxophone Colossus''
  \end{itemize}
\item
  Gene Ammons (Tenor, 1925-1974)

  \begin{itemize}
  \tightlist
  \item
    Sometimes lumped in with the Bop musicians
  \item
    Made a lot of recordings with Sonny Stitt
  \end{itemize}
\item
  Benny Golson (Tenor/Composer, 1929-Present)

  \begin{itemize}
  \tightlist
  \item
    Early in his career played in Art Blakey's group ``The Jazz
    Messengers''
  \item
    Co-led the Jazztet with Art Farmer
  \end{itemize}
\item
  Joe Henderson (Tenor/Composer, 1937-2001)

  \begin{itemize}
  \tightlist
  \item
    ``Song for my Father'' -- soloist for our recording
  \item
    Daring improviser

    \begin{itemize}
    \tightlist
    \item
      Start simple and build to longer, more complex phrases
    \end{itemize}
  \end{itemize}
\item
  John Coltrane (Tenor/Composer, 1926-1967)

  \begin{itemize}
  \tightlist
  \item
    Emerged during the Hard Bop period
  \item
    Played with Miles Davis
  \end{itemize}
\item
  J.J. Johnson (Trombone, 1924-2001)

  \begin{itemize}
  \tightlist
  \item
    Evolved from Bop to Hard Bop
  \end{itemize}
\item
  Curtis Fuller (Trombone, 1934-Present)

  \begin{itemize}
  \tightlist
  \item
    Sideman for many influential groups

    \begin{itemize}
    \tightlist
    \item
      Jazz Messenger
    \item
      Art Farmer
    \item
      Only trombonist to be a sideman with John Coltrane
    \end{itemize}
  \end{itemize}
\item
  All the following guitarists were directly influenced by the swing era
  guitarist Charlie Christian

  \begin{itemize}
  \tightlist
  \item
    Pioneer of the amplified guitar
  \item
    Arch top guitar

    \begin{itemize}
    \tightlist
    \item
      Patterend after violins and cellos
    \end{itemize}
  \end{itemize}
\item
  Many jazz musicians would play for commercial recordings under
  pseudonyms
\item
  Wes Montgomery (Guitar, 1923-1968)
\end{enumerate}

\begin{itemize}
\tightlist
\item
  Possibly the most influential guitarist since Charlie Christian
\item
  Completely self-taught

  \begin{itemize}
  \tightlist
  \item
    Could not read music
  \item
    Played with his thumb instead of a pick
  \item
    Didn't start playing until he was 33
  \end{itemize}
\item
  Melodic improviser
\item
  Developed the octave technique

  \begin{itemize}
  \tightlist
  \item
    playing notes in octaves on a guitar
  \end{itemize}
\end{itemize}

\begin{enumerate}
\def\labelenumi{\arabic{enumi}.}
\setcounter{enumi}{13}
\tightlist
\item
  Kenny Burrell (Guitar, 1931-Present)
\end{enumerate}

\begin{itemize}
\tightlist
\item
  Director of jazz studies at the University of California, Los Angeles
\item
  Arguably equal in importance to Wes Montgomery
\item
  Finished top 19 in Downbeat magazine reader's poll over 50 consecutive
  years
\item
  Influenced by Christian, blues
\item
  Side man for Dizzy Gillespie for first recording
\item
  Idols: Parker, Fillespie
\end{itemize}

\begin{enumerate}
\def\labelenumi{\arabic{enumi}.}
\setcounter{enumi}{14}
\tightlist
\item
  Joe Pass (Guitar, 1929-1994)
\end{enumerate}

\begin{itemize}
\tightlist
\item
  Solo guitar style

  \begin{itemize}
  \tightlist
  \item
    Chord melodies -- playing chords on lower strings with melodic
    elemtns on higher strings
  \end{itemize}
\end{itemize}

\begin{enumerate}
\def\labelenumi{\arabic{enumi}.}
\setcounter{enumi}{15}
\tightlist
\item
  Grant Green (Guitar, 1935-1979)
\end{enumerate}

\begin{itemize}
\tightlist
\item
  Dead tone to the strings

  \begin{itemize}
  \tightlist
  \item
    Did not allow notes to ring
  \end{itemize}
\item
  ``Grant Stand'' in the digital library
\item
  Strong sense of rhythm and syncopation
\end{itemize}

\begin{enumerate}
\def\labelenumi{\arabic{enumi}.}
\setcounter{enumi}{16}
\tightlist
\item
  Pat Martino (Guitar, 1944-Present)
\item
  George Benson (Guitar, 1943-Present)
\end{enumerate}

\begin{itemize}
\tightlist
\item
  One of the great virtuoso guitarists
\end{itemize}

\begin{enumerate}
\def\labelenumi{\arabic{enumi}.}
\setcounter{enumi}{18}
\tightlist
\item
  Jazz organ
\end{enumerate}

\begin{itemize}
\tightlist
\item
  Hammond B11

  \begin{itemize}
  \tightlist
  \item
    Had a percussive attack for each note instead of the traditional
    organ swell
  \end{itemize}
\item
  Organ trio -- Organ, Drums, Guitar
\end{itemize}

\begin{enumerate}
\def\labelenumi{\arabic{enumi}.}
\setcounter{enumi}{19}
\tightlist
\item
  Jimmy Smith (Organ, 1928-2005)
\item
  Jack McDuff (Organ, 1926-2001)
\item
  Jimmy McGriff (Organ, 1936-2008)
\item
  Tommy Flanagan (Piano, 1930-2001)
\end{enumerate}

\begin{itemize}
\tightlist
\item
  From Detroit (Same as Kenny Burrell)
\item
  Major influence was Art Tatum
\end{itemize}

\begin{enumerate}
\def\labelenumi{\arabic{enumi}.}
\setcounter{enumi}{23}
\tightlist
\item
  Art Tatum (Piano, 1909-1956)
\end{enumerate}

\begin{itemize}
\tightlist
\item
  Avoided piano keys
\end{itemize}

\begin{enumerate}
\def\labelenumi{\arabic{enumi}.}
\setcounter{enumi}{24}
\tightlist
\item
  Horace Silver (Piano, 1928-2013)
\end{enumerate}

\begin{itemize}
\tightlist
\item
  Most prolific composer of Hard Bop
\item
  Not virtuoso
\item
  ``Senor Blues''
\item
  Horace Silver quintet
\item
  Founded Jazz Messengers with Art Blakey
\end{itemize}

\begin{enumerate}
\def\labelenumi{\arabic{enumi}.}
\setcounter{enumi}{25}
\tightlist
\item
  Ahmad Jamal (Piano, 1930-Present)
\end{enumerate}

\begin{itemize}
\tightlist
\item
  Liked piano + bass + drums
\end{itemize}

\begin{enumerate}
\def\labelenumi{\arabic{enumi}.}
\setcounter{enumi}{26}
\tightlist
\item
  Clifford ``Brownie'' Brown (Trumpet, 1930-1956)
\end{enumerate}

\begin{itemize}
\tightlist
\item
  Admired widely, but not well known outside of jazz
\item
  No drugs
\item
  Ease/comfort of playing difficult pieces
\item
  ``Daahoud''

  \begin{itemize}
  \tightlist
  \item
    Goes a long time during solos without braething
  \end{itemize}
\item
  He and Richie Powell died in a car accident
\end{itemize}

\begin{enumerate}
\def\labelenumi{\arabic{enumi}.}
\setcounter{enumi}{27}
\tightlist
\item
  Freddie Hubbard (Trumpet, 1928-2009)
\end{enumerate}

\begin{itemize}
\tightlist
\item
  Influence by Clifford Brown, Chet Baker, Miles Davis
\item
  Manipulate pitch and tone depending on musical setting
\item
  Strong sense of rhythm - double time
\item
  Harmonically daring - took chances, tried things he hadn't done before
\end{itemize}

\begin{enumerate}
\def\labelenumi{\arabic{enumi}.}
\setcounter{enumi}{28}
\tightlist
\item
  Lee Morgan (Trumpet, 1938-1972)
\end{enumerate}

\begin{itemize}
\tightlist
\item
  Primary influence was Clifford Brown (took lessons from him)
\item
  Side man on many important Hard Bop recordings
\end{itemize}

\begin{enumerate}
\def\labelenumi{\arabic{enumi}.}
\setcounter{enumi}{29}
\tightlist
\item
  Max Roach (Drums, 1924-2007)
\end{enumerate}

\begin{itemize}
\tightlist
\item
  Bop -\textgreater{} Hard Bop
\end{itemize}

\begin{enumerate}
\def\labelenumi{\arabic{enumi}.}
\setcounter{enumi}{30}
\tightlist
\item
  ``Philly'' Joe Jones (Drums, 1923-1985)

  \begin{itemize}
  \tightlist
  \item
    Known for call and response
  \end{itemize}
\item
  Art Blakey (Drums, 1919-1991)

  \begin{itemize}
  \tightlist
  \item
    Less rigid timekeeping, drums for accent
  \item
    Dynamic accompianist
  \item
    Cofounded Jazz Messengers with Horace Silver
  \item
    Hand picked young musicians
  \item
    Led band musically, not visually
  \end{itemize}
\end{enumerate}

\section{Miles Davis (Trumpet/Composer/Band Leader,
1926-1991)}\label{miles-davis-trumpetcomposerband-leader-1926-1991}

\begin{enumerate}
\def\labelenumi{\arabic{enumi}.}
\tightlist
\item
  One of the `one-namers'
\item
  Member of Rock and Roll Hall of Fame
\item
  On the forefront of many jazz trends before they were popular

  \begin{itemize}
  \tightlist
  \item
    At the cost of alienating his fan base
  \end{itemize}
\item
  Relatively privileged childhood compared to most jazz musicians of the
  time

  \begin{itemize}
  \tightlist
  \item
    Son of a prominent dentist
  \end{itemize}
\item
  Grew up in Alton, Illinois

  \begin{itemize}
  \tightlist
  \item
    White neighborhood
  \end{itemize}
\item
  Went to Juilliard for college to study music

  \begin{itemize}
  \tightlist
  \item
    Did not have the intention of being a classical trumpet player
  \item
    Wanted to go to New York to be with the up and coming jazz scene
  \item
    Did not graduate, left school to join Charlie Parker's Band
  \end{itemize}
\item
  Jazz idols were Charlie Parker and Dizzy Gillespie

  \begin{itemize}
  \tightlist
  \item
    Sat in with them once while he was in high school
  \end{itemize}
\item
  First recorded with Charlie Parker's band in 1945 (1819 years old)
\item
  Created an original trumpet style unlike any of his predecessors

  \begin{itemize}
  \tightlist
  \item
    Used fewer notes
  \item
    Revolutionary use of silence
  \end{itemize}
\item
  Put out a large volume of recordings

  \begin{itemize}
  \tightlist
  \item
    The standard recording for many standards
  \end{itemize}
\item
  Collaborated with many other musicians, notably Gil Evans
  (piano/composer)

  \begin{itemize}
  \tightlist
  \item
    First in 1949, released in 1950 ``Birth of the Cool''

    \begin{itemize}
    \tightlist
    \item
      unusual instrumentation tuba, French horn
    \item
      soft but intense
    \end{itemize}
  \item
    Jazz orchestras

    \begin{itemize}
    \tightlist
    \item
      Did not play dance tunes, more like symphony orchestras with jazz
      instruments
    \end{itemize}
  \item
    ``Porgy and Bess''
  \item
    ``Miles Ahead''
  \item
    ``Sketches of Spain''
  \item
    ``Quiet Nights''
  \end{itemize}
\item
  Pioneered modal jazz

  \begin{itemize}
  \tightlist
  \item
    Jazz built on a simplified, slower harmonic progression
  \end{itemize}
\item
  Pioneered jazz fusion

  \begin{itemize}
  \tightlist
  \item
    Blend of 3 different styles -- jazz (improv), rock, funk
  \end{itemize}
\item
  Notably used Harmon mute frequently

  \begin{itemize}
  \tightlist
  \item
    Changes the sound of the horn
  \item
    Associated with Miles Davis, ``Miles Davis'' mute
  \end{itemize}
\item
  Started as a side man with Charlie Parker
\item
  Classic Quintet

  \begin{itemize}
  \tightlist
  \item
    1955-1958
  \item
    Miles Davis
  \item
    Cannonball Adderly (Alto)
  \item
    Red Garland (Piano) -\textgreater{} Later Bill Evans
  \item
    Paul Chambers (Bass)
  \item
    Philly Joe Jones (Drums)
  \item
    John Coltrane (Tenor) sometimes
  \end{itemize}
\item
  ``Kind of Blue'' - 1959

  \begin{itemize}
  \tightlist
  \item
    Classic Sextet members
  \item
    Start of modal jazz approach
  \item
    One of 2 best selling jazz album of all time with Dave Brubeck's
    ``Time Out''
  \end{itemize}
\item
  Second Great Quintet

  \begin{itemize}
  \tightlist
  \item
    1964-1969
  \item
    Miles Davis (Trumpet)
  \item
    Wayne Shorter (Tenor)
  \item
    Herbie Hancock (Piano)
  \item
    Ron Carter (Bass)
  \item
    Tony Williams (Drums)
  \item
    Closer to Avant-Garde style
  \end{itemize}
\item
  Electric Miles Period of his career

  \begin{itemize}
  \tightlist
  \item
    1969-1991
  \item
    Jazz Fusion
  \item
    Replaced traditional acoustic instruments with electric versions
  \item
    Inspired by the rock and roll groups of the time
  \item
    First jazz music not focused on horns or singing
  \item
    Recorded 15 albums in 4 years
  \item
    Accused of selling out -- music much easier to play, not as
    interactive
  \end{itemize}
\item
  Became addicted to heroin

  \begin{itemize}
  \tightlist
  \item
    Decided to kick his habit by locking himself in a room for a week
  \end{itemize}
\item
  Big fan of boxer `Sugar' Ray Robinson

  \begin{itemize}
  \tightlist
  \item
    Seeing his dedication and resolve inspired Davis to quit heroin
  \end{itemize}
\item
  Known for tough, mean attitude
\item
  Became the best paid jazz musician at the time
\item
  Feared prejudice in America

  \begin{itemize}
  \tightlist
  \item
    Was beaten by a white cop while taking a break at a club he was
    working
  \end{itemize}
\item
  When it came to music, color didn't matter to Davis
\item
  Was known to be able to bring out everyone's individuality in his
  group while remaining in his vision
\end{enumerate}

\section{John Coltrane (Tenor/Soprano/Alto,
1926-1967)}\label{john-coltrane-tenorsopranoalto-1926-1967}

\begin{enumerate}
\def\labelenumi{\arabic{enumi}.}
\tightlist
\item
  Primarily tenor
\item
  Transformative
\item
  Always evolving, improving
\item
  Laser focused tone, particularly in the upper register
\item
  Very imitated sound
\item
  Influenced by the beboppers early in his career

  \begin{itemize}
  \tightlist
  \item
    Dexter Gordon
  \item
    Sonny Stitt
  \end{itemize}
\item
  Also influenced by Lester Young
\item
  Later took inspiration from the Avant-Garde movement

  \begin{itemize}
  \tightlist
  \item
    They in turn where inspired by him
  \end{itemize}
\item
  Veteran of rhythm and blues band
\item
  Sideman with Miles Davis in 1950s (Classic Quintet)
\item
  Solos described as sheets of sound

  \begin{itemize}
  \tightlist
  \item
    Flurry of notes
  \end{itemize}
\item
  Band leader

  \begin{itemize}
  \tightlist
  \item
    Still sideman for Miles Davis
  \item
    First album ``Blue Train'' - 1967

    \begin{itemize}
    \tightlist
    \item
      Lee Morgan (Trumpet)
    \end{itemize}
  \item
    Compositions known for rapidly changing harmonic progressions

    \begin{itemize}
    \tightlist
    \item
      ``Moment's Notice'' -- named because pianist during recording
      proclaimed ``You can't expect us to play this at a moment's
      notice''
    \end{itemize}
  \item
    ``Giant Steps''

    \begin{itemize}
    \tightlist
    \item
      Tommy Flanagan (Piano)
    \end{itemize}
  \end{itemize}
\item
  Re-popularized soprano

  \begin{itemize}
  \tightlist
  \item
    Popular in the swing era
  \item
    Unused in bop
  \end{itemize}
\item
  ``A Love Supreme''

  \begin{itemize}
  \tightlist
  \item
    LP -- Only 4 songs
  \item
    ``Resolution'' - Final exam listening list

    \begin{itemize}
    \tightlist
    \item
      Popular among jazz musicians
    \end{itemize}
  \end{itemize}
\item
  Classic Quartet

  \begin{itemize}
  \tightlist
  \item
    McCoy Tyner (Piano)
  \item
    Jimmy Garrison (Bass)
  \item
    Elvin Jones (Drums)
  \end{itemize}
\item
  Died of Cancer at 40 years old
\item
  Only had 12 years of recorded work
\end{enumerate}

\end{document}
