\documentclass[]{article}
\usepackage{lmodern}
\usepackage{amssymb,amsmath}
\usepackage{ifxetex,ifluatex}
\usepackage{fixltx2e} % provides \textsubscript
\ifnum 0\ifxetex 1\fi\ifluatex 1\fi=0 % if pdftex
  \usepackage[T1]{fontenc}
  \usepackage[utf8]{inputenc}
\else % if luatex or xelatex
  \ifxetex
    \usepackage{mathspec}
  \else
    \usepackage{fontspec}
  \fi
  \defaultfontfeatures{Ligatures=TeX,Scale=MatchLowercase}
\fi
% use upquote if available, for straight quotes in verbatim environments
\IfFileExists{upquote.sty}{\usepackage{upquote}}{}
% use microtype if available
\IfFileExists{microtype.sty}{%
\usepackage{microtype}
\UseMicrotypeSet[protrusion]{basicmath} % disable protrusion for tt fonts
}{}
\usepackage{hyperref}
\hypersetup{unicode=true,
            pdfborder={0 0 0},
            breaklinks=true}
\urlstyle{same}  % don't use monospace font for urls
\IfFileExists{parskip.sty}{%
\usepackage{parskip}
}{% else
\setlength{\parindent}{0pt}
\setlength{\parskip}{6pt plus 2pt minus 1pt}
}
\setlength{\emergencystretch}{3em}  % prevent overfull lines
\providecommand{\tightlist}{%
  \setlength{\itemsep}{0pt}\setlength{\parskip}{0pt}}
\setcounter{secnumdepth}{0}
% Redefines (sub)paragraphs to behave more like sections
\ifx\paragraph\undefined\else
\let\oldparagraph\paragraph
\renewcommand{\paragraph}[1]{\oldparagraph{#1}\mbox{}}
\fi
\ifx\subparagraph\undefined\else
\let\oldsubparagraph\subparagraph
\renewcommand{\subparagraph}[1]{\oldsubparagraph{#1}\mbox{}}
\fi

\date{}

\begin{document}

\section{General Knowledge}\label{general-knowledge}

\begin{enumerate}
\def\labelenumi{\arabic{enumi}.}
\tightlist
\item
  harmony = chord progression = chord changes = chords = changes
\item
  Jazz musicians tend to not follow a particular movement, but rather
  their own interests

  \begin{itemize}
  \tightlist
  \item
    They tend to dislike labels assigned to them
  \end{itemize}
\item
  Side = Individual song on a record
\end{enumerate}

\section{Bop (1940s - Present)}\label{bop-1940s---present}

\begin{enumerate}
\def\labelenumi{\arabic{enumi}.}
\tightlist
\item
  Followed the Swing Era
\item
  Was a rebellion against the commercialization of jazz that had
  occurred
\item
  Started with jam sessions where musicians could branch out and be more
  creative
\item
  Combos more popular than big bands
\item
  Charlie Parker (Alto, 1920-1955)

  \begin{itemize}
  \tightlist
  \item
    Most influential Bop musician
  \item
    Unique approach to improvisation

    \begin{itemize}
    \tightlist
    \item
      Used dissonance well
    \end{itemize}
  \item
    From Kansas City, Missouri (town known for blues)
  \item
    Married at 16 (soon divorced to move to NY)
  \item
    Became addicted to heroin at 17
  \item
    Known as ``Bird''
  \item
    Later in life was committed to a mental hospital after passing out
    and setting his bed on fire
  \item
    After getting out of the hospital, he formed a quartet containing
    Miles Davis
  \item
    Hated the term ``Bebop'' -- ``It's just music''
  \item
    Very critical of his own work and shy about the praise he received
  \item
    Loved all kinds of music

    \begin{itemize}
    \tightlist
    \item
      Known to listen to Country music
    \end{itemize}
  \end{itemize}
\item
  John Birks (Dizzy) Gillespie (Trumpet, 1917-1993)

  \begin{itemize}
  \tightlist
  \item
    Most influential Bop trumpet player
  \item
    Known for great range
  \item
    Name comes from his unpredictability

    \begin{itemize}
    \tightlist
    \item
      Would often dance during others' solos
    \end{itemize}
  \item
    Public face of Bop
  \item
    Broke many barriers

    \begin{itemize}
    \tightlist
    \item
      Hired a female trombonist and a Cuban bongo player for his band
    \end{itemize}
  \item
    Tied in jazz and Caribbean music
  \item
    Tried to make Bop accessible to everyone

    \begin{itemize}
    \tightlist
    \item
      Failed to attract dancers
    \end{itemize}
  \end{itemize}
\item
  Thelonius Monk (Piano, 1917-1982)

  \begin{itemize}
  \tightlist
  \item
    Not very technically proficient
  \item
    Unique harmonic approach
  \item
    Embraced dissonance
  \item
    ``Logical''
  \item
    Resident pianist at Minton's
  \item
    Eccentric: Would dress oddly, speak very little, get up during a
    performance to dance

    \begin{itemize}
    \tightlist
    \item
      caused critics to dislike/dismiss him
    \end{itemize}
  \item
    Denied a cabaret card after refusing to testify against Bud Powell
    for possession of narcotics

    \begin{itemize}
    \tightlist
    \item
      Means he could not play in venues that sold alcohol
    \item
      Spent 6 years writing music
    \item
      Released an album that received great reviews which sparked new
      popularity for him
    \end{itemize}
  \end{itemize}
\item
  Tadd Dameron (Piano/Composer, 1917-1965)

  \begin{itemize}
  \tightlist
  \item
    Wrote many modern day jazz standards
  \item
    Wrote Hot House (Our recording is by Dizzy Gillespie)
  \item
    Led larger groups (7-10 musicians)
  \end{itemize}
\item
  John Lewis (Piano/Composer, 1920-2001)

  \begin{itemize}
  \tightlist
  \item
    Spanned many periods
  \item
    Modern Jazz Quartet founder (Late in Bop, persisted through Cool)

    \begin{itemize}
    \tightlist
    \item
      Former members of Dizzy Gillespie's band
    \item
      Played in concert halls and wore tuxedos
    \item
      Piano, Drums, Bass, and Vibes
    \end{itemize}
  \item
    Loathed the dissipation and drug use of Charlie Parker
  \item
    Insisted that his music be played with dignity
  \end{itemize}
\item
  Kenny Clarke (Drums, 1914-1985)

  \begin{itemize}
  \tightlist
  \item
    Changed the way that drummers played, invented Bop style
  \item
    Drummer in house band at Minton's Playhouse
  \item
    Kept time on the cymbal
  \end{itemize}
\item
  Minton's Playhouse

  \begin{itemize}
  \tightlist
  \item
    Location for after hours jam sessions that gave rise to the Bop
    period
  \end{itemize}
\item
  Max Roach (Drums, 1924-2007)

  \begin{itemize}
  \tightlist
  \item
    Also known for hard Bop period
  \end{itemize}
\item
  Charles Mingus (Bass/Composer, 1922-1979)

  \begin{itemize}
  \tightlist
  \item
    Spanned multiple periods
  \end{itemize}
\item
  Ray Brown (Bass, 1926-2002)
\item
  Oscar Pettiford (Bass/Cello, 1922-1960)
\item
  Bud Powell (Piano, 1924-1966)

  \begin{itemize}
  \tightlist
  \item
    Brought Bop to the Keyboard
  \item
    ``Could outbird Bird and outdizzy Dizzy''
  \item
    Debatably a better improviser than Charlie Parker
  \end{itemize}
\item
  George Shearing (Piano, 1919-2011)

  \begin{itemize}
  \tightlist
  \item
    Commercially popular
  \item
    Worked with Mel Torme
  \end{itemize}
\item
  Oscar Peterson (Piano, 1925-2007)

  \begin{itemize}
  \tightlist
  \item
    One of the most admired pianists ever
  \item
    Incredibly technical
  \end{itemize}
\item
  Dexter Gordon (Tenor, 1923-1990)

  \begin{itemize}
  \tightlist
  \item
    Bop Pioneer for tenor sax
  \end{itemize}
\item
  Stan Getz (Tenor, 1927-1991)

  \begin{itemize}
  \tightlist
  \item
    Spanned multiple periods

    \begin{itemize}
    \tightlist
    \item
      Cool, Bossa Nova
    \end{itemize}
  \item
    Member of Woody Herman's Thundering Herd
  \end{itemize}
\item
  Sonny Stitt (Alto/Tenor, 1924-1982)

  \begin{itemize}
  \tightlist
  \item
    Criticized for copying Charlie Parker
  \end{itemize}
\item
  Fats Navarro (Trumpet, 1923-1950)
\item
  J.J. Johnson (Trombone, 1924-2001)
\item
  Woody Herman's Thundering Herd

  \begin{itemize}
  \tightlist
  \item
    ``Four Brothers''

    \begin{itemize}
    \tightlist
    \item
      Named for the 4 saxophone players in the band
    \end{itemize}
  \item
    Big Band with some Bop music
  \end{itemize}
\item
  Primarily developed in New York
\item
  Primarily African-American
\end{enumerate}

\section{Cool (Late 1940s - Present)}\label{cool-late-1940s---present}

\begin{enumerate}
\def\labelenumi{\arabic{enumi}.}
\tightlist
\item
  Easier to follow and softer than Bop
\item
  May be viewed as a reaction to Bop (Harrison doesn't think so though)
\item
  Primarily developed in L.A./west coast (vs.~NY for Bop)
\item
  Primarily Caucasian (vs.~African-American for Bop)
\item
  Synonymous with ``West Coast Sound''

  \begin{itemize}
  \tightlist
  \item
    Really West Coast Sound contained in Cool
  \end{itemize}
\item
  Dry, light, airy sound compared to harsher, stronger Bop sound
\item
  Lester Young (Tenor, 1909-1959)

  \begin{itemize}
  \tightlist
  \item
    Influenced both Bop and Cool
  \item
    Known as a great swing era improviser
  \item
    Played with Basie
  \item
    Played at Minton's
  \item
    Used dissonance to create new harmonies in solos (influenced Bop)
  \item
    Legato phrasing (influenced Cool)
  \item
    Uncharacteristic sound

    \begin{itemize}
    \tightlist
    \item
      Most other tenors at the time gravitated to the lower register
    \item
      Lester approached it more like an Alto player would
    \item
      Delicate, light
    \end{itemize}
  \end{itemize}
\item
  Miles Davis (Trumpet, 1926-1991)

  \begin{itemize}
  \tightlist
  \item
    Disliked the Cool label
  \item
    New York musician
  \item
    Album: ``Birth of the Cool'' (1951)

    \begin{itemize}
    \tightlist
    \item
      Often mistakenly assumed to be the start of Cool
    \item
      First of many collaborations with composer/pianist Ernest Gilmore
      ``Gil'' Evans
    \item
      10 piece group
    \item
      Notably contained tuba, bass clarinet, french horn, flute
    \item
      Lee Konitz as a side man
    \end{itemize}
  \end{itemize}
\item
  Stan Getz (Tenor, 1927-1991)

  \begin{itemize}
  \tightlist
  \item
    Disliked the Cool label
  \end{itemize}
\item
  Modern Jazz Quartet (MJQ)

  \begin{itemize}
  \tightlist
  \item
    Disliked the Cool label
  \end{itemize}
\item
  Lennie Tristano (Composer/Band Leader/Piano, 1919-1978)

  \begin{itemize}
  \tightlist
  \item
    Largely overlooked
  \item
    One of the main creators of Cool as a modern jazz alternative to Bop
  \item
    Influenced by pianist Art Tatum and Lester Young
  \item
    Also influence by Bach
  \item
    Less ``jumpy'' than Bop, smoother

    \begin{itemize}
    \tightlist
    \item
      Not necessarily melodic though
    \end{itemize}
  \item
    Often overlooked because he did not release many records
  \item
    Employed collective improvisation

    \begin{itemize}
    \tightlist
    \item
      Multiple soloists at once
    \end{itemize}
  \item
    Influence on Bill Evans (piano)
  \item
    Nearly blind
  \item
    Played with splayed fingers on right hand, but curved left
  \end{itemize}
\item
  Lee Konitz (Alto, 1927-Present)

  \begin{itemize}
  \tightlist
  \item
    Was a student of Lennie Tristano

    \begin{itemize}
    \tightlist
    \item
      Later colleague and band mate
    \end{itemize}
  \item
    Some claim he could rival Charlie Parker

    \begin{itemize}
    \tightlist
    \item
      Very different sound
    \item
      Konitz was dry, airy, light vs.~Parker's brittle sound
    \end{itemize}
  \item
    Preferred upper register
  \end{itemize}
\item
  Big bands had mostly disappeared by this time

  \begin{itemize}
  \tightlist
  \item
    Some survivor's were Woody Herman's and Stan Kenton's bands

    \begin{itemize}
    \tightlist
    \item
      Both moved to the West Coast
    \item
      Many noteworthy musicians started as members of these bands before
      contonuing on to solo careers
    \end{itemize}
  \item
    Transition from dance bands to performance orchestra for a seated
    audience

    \begin{itemize}
    \tightlist
    \item
      More features and solos
    \item
      More tempo changes (taboo for dance bands)
    \end{itemize}
  \end{itemize}
\item
  Warne Marsh (Tenor, 1927-1987)

  \begin{itemize}
  \tightlist
  \item
    Protégé of Lee Konitz
  \end{itemize}
\item
  Chet Baker (Trumpet/Vocalist, 1929-1988)

  \begin{itemize}
  \tightlist
  \item
    Notably did not start in Woody Herman or Stan Kenton's band
  \item
    Started with Jerry Mulligan's (Bari sax player) quartet
  \end{itemize}
\item
  Shorty Rogers (Trumpet/Composer, 1924-1994)

  \begin{itemize}
  \tightlist
  \item
    Played with and wrote for both Woody Herman and Stan Kenton
  \end{itemize}
\item
  Jack Sheldon (Trumpet/Vocalist, 1931-Present)

  \begin{itemize}
  \tightlist
  \item
    Also an actor
  \item
    Sang for School House Rock (``I'm Just a Bill'' among others)
  \end{itemize}
\item
  Art Farmer (Trumpet/Flugelhorn, 1928-1999)

  \begin{itemize}
  \tightlist
  \item
    Originally from Iowa
  \item
    Early teens in Phoenix
  \item
    Started with Lionel Hampton
  \end{itemize}
\item
  Art Pepper (Alto, 1925-1982)

  \begin{itemize}
  \tightlist
  \item
    Started with Stan Kenton
  \end{itemize}
\item
  Jimmy Giuffre (Tenor/Clarinet/Composer, 1921-2008)

  \begin{itemize}
  \tightlist
  \item
    Wrote for Woody Herman

    \begin{itemize}
    \tightlist
    \item
      Wrote ``Four Brothers''
    \end{itemize}
  \end{itemize}
\item
  Paul ``Dry Martini'' Desmond (Alto, 1924-1977)

  \begin{itemize}
  \tightlist
  \item
    Notably did not play with Woody Herman or Stan Kenton
  \item
    Played in Dave Brubeck quartet (only horn player)
  \item
    Similar sound to Lee Konitz
  \end{itemize}
\item
  Gerry Mulligan (Bari/Composer, 1927-1996)

  \begin{itemize}
  \tightlist
  \item
    Sideman on Miles Davis' ``Birth of the Cool''

    \begin{itemize}
    \tightlist
    \item
      As a result, offered a residency at an L.A. club causing him to
      move from New York
    \end{itemize}
  \item
    Uncommon combo instrument
  \item
    Brought the lighter sound of Cool to the Bari
  \item
    Jerry Mulligan's Quartet

    \begin{itemize}
    \tightlist
    \item
      Notably did not have a piano (club probably didn't have one)
    \item
      Trumpet, Bari, Bass, Drums
    \end{itemize}
  \end{itemize}
\item
  Chico Hamilton (Drums, 1921-2013)

  \begin{itemize}
  \tightlist
  \item
    One of the relatively few African-American Cool musicians
  \item
    Played for Stan Kenton's Big Band and Gerry Mulligan's Quartet
  \item
    Ventured away from the ride rhythm
  \end{itemize}
\item
  Shelly Manne (Drums, 1921-2008)

  \begin{itemize}
  \tightlist
  \item
    Played with both Stan Kenton and Woody Herman
  \item
    Drummer for Henry Mancini (TV and Film Composer) (Pink Panther and
    more)
  \end{itemize}
\item
  Jim Hall (Guitar, 1930-2013)

  \begin{itemize}
  \tightlist
  \item
    Mellow sound
  \item
    Did not play very technically
  \end{itemize}
\item
  Carl Fontana (Trombone, 1928-2003)

  \begin{itemize}
  \tightlist
  \item
    Played with Woody Herman
  \item
    Considered second only to J.J. Johnson
  \item
    Could be considered a Bop musician
  \end{itemize}
\item
  Frank Rosolino (Trombone, 1926-1978)

  \begin{itemize}
  \tightlist
  \item
    Could be considered a Bop musician
  \item
    The top west coast trombone player
  \item
    Played with Stan Kenton

    \begin{itemize}
    \tightlist
    \item
      Some compositions were named after him (``Frank Speaking'')
    \end{itemize}
  \end{itemize}
\item
  Bob Brookmeyer (Trombone/Composer, 1929-2011)

  \begin{itemize}
  \tightlist
  \item
    Valve trombone

    \begin{itemize}
    \tightlist
    \item
      Easy way out to avoid the difficulties of a slide
    \end{itemize}
  \end{itemize}
\item
  Dave Brubeck (Piano/Composer, 1920-2012)

  \begin{itemize}
  \tightlist
  \item
    Leader of Dave Brubeck quartet
  \item
    Most commercially successful Cool pianist
  \item
    Popular among college students

    \begin{itemize}
    \tightlist
    \item
      Toured college campuses
    \item
      ``Jazz Goes to College'' album
    \end{itemize}
  \item
    ``Time Out'' (our listening list)

    \begin{itemize}
    \tightlist
    \item
      Unusual time signatures
    \item
      Take 5 (5/4)
    \item
      Blue Rondo a la Turk (9/8)

      \begin{enumerate}
      \def\labelenumii{\roman{enumii}.}
      \tightlist
      \item
        Blue -\textgreater{} Early days of jazz
      \item
        Rondo -\textgreater{} Form of the song, alternating sessions but
        always returning to the original. ABACAD\ldots{}
      \item
        a la Turk -\textgreater{} Tribute to Turkish musicians (Brubeck
        visited Turkey)
      \end{enumerate}
    \end{itemize}
  \end{itemize}
\end{enumerate}

\section{Bossa Nova (Popular
1958-1963)}\label{bossa-nova-popular-1958-1963}

\begin{enumerate}
\def\labelenumi{\arabic{enumi}.}
\tightlist
\item
  Incorporation of Brazilian style
\item
  ``Bossa Nova'' = ``New Trend''/``New Beat''
\item
  Antonio Carlos Jobim (Composer, 1927-1994)

  \begin{itemize}
  \tightlist
  \item
    Most popular Bossa Nova composer
  \item
    Worked with Stan Getz
  \item
    Very important in bringing Bossa Nova to the U.S.
  \item
    ``Desafinado''

    \begin{itemize}
    \tightlist
    \item
      From album ``Getz/Gilberto''
    \end{itemize}
  \end{itemize}
\end{enumerate}

\section{Hard Bop (Mid 1950s-Present)}\label{hard-bop-mid-1950s-present}

\begin{enumerate}
\def\labelenumi{\arabic{enumi}.}
\tightlist
\item
  Julian ``Cannonball'' Adderly (Alto, 1928-1975)

  \begin{itemize}
  \tightlist
  \item
    Considered most influential saxophonist since Charlie Parker
  \item
    Powerful but fluid sound
  \item
    Sideman with Miles Davis
  \item
    Later co-ran Cannonball Adderly Quartet with his brother Nat Adderly
  \end{itemize}
\item
  Sonny Rollins (Tenor/Composer, 1930-Present)

  \begin{itemize}
  \tightlist
  \item
    The titan of jazz
  \item
    Initially influenced by Charlie Parker
  \item
    Stopped performing in the late 60s because he felt that he was not
    growing as a musician
  \item
    Isolated himself
  \item
    Would go to the Williamsburg bridge in New York to practice

    \begin{itemize}
    \tightlist
    \item
      Artistic rediscovery
    \item
      Released ``The Bridge''

      \begin{enumerate}
      \def\labelenumii{\roman{enumii}.}
      \tightlist
      \item
        ``Without a Song'' on our listening list
      \end{enumerate}
    \end{itemize}
  \item
    Became addicted to heroin

    \begin{itemize}
    \tightlist
    \item
      Worked as a day laborer for a year to get himself clean
    \end{itemize}
  \item
    ``Saxophone Colossus''
  \end{itemize}
\item
  Gene Ammons (Tenor, 1925-1974)

  \begin{itemize}
  \tightlist
  \item
    Sometimes lumped in with the Bop musicians
  \item
    Made a lot of recordings with Sonny Stitt
  \end{itemize}
\item
  Benny Golson (Tenor/Composer, 1929-Present)

  \begin{itemize}
  \tightlist
  \item
    Early in his career played in Art Blakey's group ``The Jazz
    Messengers''
  \item
    Co-led the Jazztet with Art Farmer
  \end{itemize}
\item
  Joe Henderson (Tenor/Composer, 1937-2001)

  \begin{itemize}
  \tightlist
  \item
    ``Song for my Father'' -- soloist for our recording
  \item
    Daring improviser

    \begin{itemize}
    \tightlist
    \item
      Start simple and build to longer, more complex phrases
    \end{itemize}
  \end{itemize}
\item
  John Coltrane (Tenor/Composer, 1926-1967)

  \begin{itemize}
  \tightlist
  \item
    Emerged during the Hard Bop period
  \item
    Played with Miles Davis
  \end{itemize}
\item
  J.J. Johnson (Trombone, 1924-2001)

  \begin{itemize}
  \tightlist
  \item
    Evolved from Bop to Hard Bop
  \end{itemize}
\item
  Curtis Fuller (Trombone, 1934-Present)

  \begin{itemize}
  \tightlist
  \item
    Sideman for many influential groups

    \begin{itemize}
    \tightlist
    \item
      Jazz Messenger
    \item
      Art Farmer
    \item
      Only trombonist to be a sideman with John Coltrane
    \end{itemize}
  \end{itemize}
\item
  All the following guitarists were directly influenced by the swing era
  guitarist Charlie Christian

  \begin{itemize}
  \tightlist
  \item
    Pioneer of the amplified guitar
  \item
    Arch top guitar

    \begin{itemize}
    \tightlist
    \item
      Patterend after violins and cellos
    \end{itemize}
  \end{itemize}
\item
  Many jazz musicians would play for commercial recordings under
  pseudonyms
\item
  Wes Montgomery (Guitar, 1923-1968)

  \begin{itemize}
  \tightlist
  \item
    Possibly the most influential guitarist since Charlie Christian
  \item
    Completely self-taught

    \begin{itemize}
    \tightlist
    \item
      Could not read music
    \item
      Played with his thumb instead of a pick
    \item
      Didn't start playing until he was 19
    \end{itemize}
  \item
    Melodic improviser
  \item
    Developed the octave technique

    \begin{itemize}
    \tightlist
    \item
      playing notes in octaves on a guitar
    \end{itemize}
  \end{itemize}
\item
  Kenny Burrell (Guitar, 1931-Present)

  \begin{itemize}
  \tightlist
  \item
    Director of jazz studies at the University of California, Los
    Angeles
  \item
    Arguably equal in importance to Wes Montgomery
  \item
    Finished top 5 in Downbeat magazine reader's poll over 50
    consecutive years
  \item
    Influenced by Christian, blues
  \item
    side man for Dizzy Gillespie for first recording
  \end{itemize}
\item
  Joe Pass (Guitar, 1929-1994)

  \begin{itemize}
  \tightlist
  \item
    Solo guitar style

    \begin{itemize}
    \tightlist
    \item
      Chord melodies -- playing chords on lower strings with melodic
      elemtns on higher strings
    \end{itemize}
  \end{itemize}
\item
  Grant Green (Guitar, 1935-1979)

  \begin{itemize}
  \tightlist
  \item
    Dead tone to the strings

    \begin{itemize}
    \tightlist
    \item
      Did not allow notes to ring
    \end{itemize}
  \item
    ``Grant Stand'' in the digital library
  \item
    Strong sense of rhythm and syncopation
  \end{itemize}
\item
  Pat Martino (Guitar, 1944-Present)
\item
  George Benson (Guitar, 1943-Present)

  \begin{itemize}
  \tightlist
  \item
    One of the great virtuoso guitarists
  \end{itemize}
\item
  Jazz organ

  \begin{itemize}
  \tightlist
  \item
    Hammond B-3

    \begin{itemize}
    \tightlist
    \item
      Had a percussive attack for each note instead of the traditional
      organ swell
    \end{itemize}
  \item
    Organ trio -- Organ, Drums, Guitar
  \end{itemize}
\item
  Jimmy Smith (Organ, 1928-2005)
\item
  Jack McDuff (Organ, 1926-2001)
\item
  Jimmy McGriff (Organ, 1936-2008)
\item
  Miles Davis (Trumpet, )
\end{enumerate}

\end{document}
